% Diese Zeile bitte -nicht- aendern.
\documentclass[course=erap]{aspdoc}

%%%%%%%%%%%%%%%%%%%%%%%%%%%%%%%%%
%% TODO: Ersetzen Sie in den folgenden Zeilen die entsprechenden -Texte-
%% mit den richtigen Werten.
\newcommand{\theGroup}{251} % Beispiel: 42
\newcommand{\theNumber}{A319} % Beispiel: A123
\author{Sina Mozaffari Tabar \and Alireza Kamalidehghan  \and Mostafa Nejati Hatamian}
\date{Sommersemester 2023} % Beispiel: Wintersemester 2019/20
%%%%%%%%%%%%%%%%%%%%%%%%%%%%%%%%%

% Diese Zeile bitte -nicht- aendern.
\title{Gruppe \theGroup{} -- Abgabe zu Aufgabe \theNumber}

\begin{document}
\maketitle

\section{Einleitung}
Die Arithmetik ist eine fundamentale Disziplin der Mathematik, die sich mit den Grundoperationen wie Addition, Subtraktion, Multiplikation und Division befasst. Normalerweise sind wir es gewohnt, diese Operationen in unserem alltäglichen Leben in einem dezimalen Zahlensystem durchzuführen, bei dem die Basis 10 beträgt.\\

Darüber hinaus die Basen 2 und 16 sind Zahlensysteme, die in der Informatik und Mathematik häufig verwendet werden. Jedes Zahlensystem basiert auf einer bestimmten Anzahl von Symbolen, die verwendet werden, um Zahlen darzustellen. Hier ist eine Erklärung, wofür wir die Basen 2 und 16 im Allgemeinen brauchen:\\

Basis 2 (Binärsystem):
Das Binärsystem verwendet nur zwei Symbole, normalerweise 0 und 1. Es ist das grundlegendste Zahlensystem in der digitalen Welt, da Computerinformationen auf zwei Zuständen basieren: ausgeschaltet (0) und eingeschaltet (1).\\

Basis 16 (Hexadezimalsystem):
Das Hexadezimalsystem verwendet 16 Symbole: die Zahlen 0-9 und die Buchstaben A-F. Es bietet eine kompaktere Darstellung großer Binärzahlen und erleichtert die Lesbarkeit und Handhabung von Zahlen in der Informatik.\\
 
 Es gibt eine allgemeine Formel,die eine Allgemiene Repräsentation von Zahlen in anderen Zahlensystemen in Dezimalschreibweise darstellt. Die Formel lautet:

$$A = \sum_{i=0} ^{n-1} a_i * g^i$$

In dieser Formel steht A für die resultierende Zahl im Dezimalsystem, n für die Anzahl der Stellen unserer Zahl, g für die Basis und $a_i$ für die i-te Ziffer der ursprünglichen Zahl.\\

Um ein besseres Verständnis zu bekommen gehen wir jetzt eine Beispiel ein. wir betrachten uns die Zahl "101011"\space in Binärsystem und mit der oben gennanten Formel wollen wir diese Zahl in Dezimalsystem umwandeln: 

\[1* 2^0 + 1*2^1 + 0* 2^2 + 1* 2^3 + 0* 2^4 + 1* 2^5 = 1+2+8+32 = 43 \]



\section{Lösungsansatz}


% TODO: Je nach Aufgabenstellung einen der Begriffe wählen
\section{Korrektheit/Genauigkeit}


\section{Performanzanalyse}


\section{Zusammenfassung und Ausblick}

% TODO: Fuegen Sie Ihre Quellen der Datei Ausarbeitung.bib hinzu
% Referenzieren Sie diese dann mit \cite{}.
% Beispiel: CR2 ist ein Register der x86-Architektur~\cite{intel2017man}.
\bibliographystyle{plain}
\bibliography{Ausarbeitung}{}

\end{document}